\section{Introduction}
Given a 2D triangular mesh $\widehat{\mathcal{M}}$ of the unit disk $\mathbb{D}$, whose triangles are all nicely shape-regular (in the sense that the ratio of their diameter to the radius of the largest inscribed circle is uniformly bounded for all triangles), we are guaranteed the existence of a conformal mapping $\Phi$ from $\mathbb{D}$ to any simply connected bounded domain $\Omega$ by the Riemann mapping theorem. This allows for a sufficiently "nice" mesh $\mathcal{M}$ on $\Omega$ to be obtained as the image of $\widehat{\mathcal{M}}$ under $\Phi$, i.e. $\mathcal{M}=\Phi(\widehat{\mathcal{M}})$. 
\bigskip

In this article we study the problem of numerically constructing conformal mappings, given a mesh on the disk and the target domain by its boundary.
We first review the theoretical background of conformal mappings and present a proof of the famous Riemann Mapping Theorem.
We then review several existing numerical methods for approximating such mappings.
The methods are compared in terms of accuracy of the approximation, computational complexity and efficiency of point evaluations for both the mapping $\Phi$ itself and its derivative (Jacobian) $D\Phi$. 
\bigskip

Finally, \red{Algorithm Name} is implemented in Python.
The algorithm is tested on several target domains and the results are discussed.