\section{Theoretical Background}
\subsection{Conformal Mappings}
\subsection{Riemann Mapping Theorem}
\subsection{Riemann-Hilbert Problem}
\subsection{Hilbert Transform}

The \textbf{Riemann mapping theorem} states that if $U$ is a non-empty simply connected open subset of the complex number plane $\mathbb{C}$ which is not all of $\mathbb{C}$, then there exists a biholomorphic mapping $f$ (i.e. a bijective holomorphic mapping whose inverse is also holomorphic) from $U$ onto the open unit disk
$$ D = \{z \in \mathbb{C}: |z|<1 \}. $$
This mapping is known as a Riemann mapping.

The beauty of the Riemann mapping theorem lies in its weight of implications, i.e. the fact that it guarantees the existence of a conformal map between any two simply connected domains in the complex plane, provided they are not the entire plane. The existence of this Riemann map is a priori not obvious: Even relatively simple Riemann mappings (for example a map from the interior of a circle to the interior of a square) have no explicit formula using only elementary functions.
Simply connected open sets in the plane can be highly complicated, for instance, the boundary can be a nowhere-differentiable fractal curve of infinite length, even if the set itself is bounded. One such example is the Koch curve.[8] The fact that such a set can be mapped in an angle-preserving manner to the nice and regular unit disc seems counter-intuitive.

\bigskip
\begin{center}
\begin{tikzpicture}[scale=0.8]
    \pgfdeclarelindenmayersystem{Koch curve}{
        \rule{F -> F-F++F-F}
        }
        \draw [black, xshift=6.7cm]
    [l-system={Koch curve, step=1.4pt, angle=60, axiom=F++F++F, order=4}]
    lindenmayer system -- cycle;

    \pgfmathsetmacro{\kochwidth}{2pt*4*8}
    \draw [black] (2,1.15) circle (\kochwidth pt);
    \node at (2,1.15) {$\disk$};
    \node at (8.7,1.15) {$\Omega$};

    \draw [->, thick] (5,1.3) -- (6,1.3) node [midway, above] {$\psi$};
\end{tikzpicture}
\end{center}
\bigskip


\begin{definition}
    A conformal mapping, also called a conformal map, conformal transformation, angle-preserving transformation, or biholomorphic map, is a transformation w=f(z) that preserves local angles. An analytic function is conformal at any point where it has a nonzero derivative.
\end{definition}


Wegmann proved the following result.

\textbf{THEOREM} When the region $G$ can be enclosed in a rectangle with sides $a$ and $b$, $b \leq a$, such that $G$ touches both small sides (see Figure 4) then the conformal mapping $\phi : D \rightarrow G$ satisfies
$$
\|\phi'\|_D \geq b \psi(b/a)
$$
with a function $\psi(\tau)$ which behaves for small $\tau$ like
$$
\psi(\tau) \approx \frac{1}{2\pi \sqrt{\epsilon}} \exp\left(\frac{\pi}{2\tau}\right).
$$

Crowding is cumbersome for all methods which work with grid points. On the other hand, methods which approximate the mapping functions by polynomials also face severe problems when the target region is elongated. 
It follows that, for the mapping of the disk to a region of aspect ratio 1 910 by a polynomial, the degree must be of several millions.
In any case, the number of grid points and the degree of the approximating polynomials increase both like exp(zr/2r) as the aspect ratio, r, tends to zero.
DeLillo [37] has shown how crowding affects the accuracy of numerical computations.
Crowding also limits the practical usefulness of conformal maps. This was demonstrated by DeLillo [36] for the Laplace equation. Crowding has also been observed for regions with elongated sections ("fingers"). For "pinched" regions, such as the interior of an inverted ellipse, ill conditioning occurs of a less severe, algebraic nature (DeLillo [37]).