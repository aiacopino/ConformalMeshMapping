\section{Theoretical Background}
%maybe change structure to section existence of a conformal map, section construction of a conformal map
\red{THROUGHOUT THIS PAPER, WE ASSUME TO BE GIVEN A SIMPLE CONNECTED REGION $\Omega\subset\C$ GIVEN BY ITS BOUNDARY $\del\Omega$ WHICH IS PARAMETRIZED BY A DIFFERENTIABLE $2\pi$-PERIODIC COMPLEX FUNCTION $\eta(s)$ WITH CONTINUOUS NON-VANISHING DERIVATIVE.}
\subsection{Conformal Mappings}

\begin{definition} \label{def:ConformalMap}
    A \textbf{conformal mapping}, also called a conformal map, conformal transformation, angle-preserving transformation, or biholomorphic map, is a transformation $f(z)$ that preserves local angles. An analytic function is conformal at any point where it has a nonzero derivative.
\end{definition}

\begin{definition}
    A \textbf{complex analytic function} or \textbf{holomorphic function} on an open subset $U\subset\C$ is locally given by a convergent power series, i.e. for any $x_0\in U$ there exists a neighborhood $V\subset U$ of $x_0$ such that for all $z\in V$ we have $$ f(z) = \sum_{n=0}^{\infty} a_n (z - z_0)^n $$ for some complex coefficients $a_n \in \C$.
    The terms analytic and holomorphic are used interchangeably.
\end{definition}

% This type of mapping is useful as some mesh properties (in particular angles) remain regular under such transformations. This ensures that cells do not become too stretched or overlap, which would cause numerical issues or even solver failure as experienced by Wechsung \cite{Wechsung2019}. 
\iffalse
\begin{definition}
    \red{Ahlfors map} too general, not needed
\end{definition}
\fi

\subsection{Riemann Mapping Theorem}
\begin{definition}
    We call a subset $\Omega \in \mathbb{C}$ \textbf{proper} if $\emptyset \neq \Omega \neq \mathbb{C}$.
\end{definition}

\begin{theorem}[Riemann Mapping Theorem]\label{thm:RiemannMappingTheorem}
    If $\Omega$ is a non-empty simply connected open proper subset of the complex plane $\mathbb{C}$, then there exists a biholomorphic mapping $f$ (i.e. a bijective holomorphic mapping whose inverse is also holomorphic) from $\Omega$ onto the open unit disk
    $$ D = \{z \in \mathbb{C}: |z|<1 \}. $$
    This mapping is known as a Riemann mapping.
\end{theorem}

The beauty of the Riemann mapping theorem lies in its weight of implications, i.e. the fact that it guarantees the existence of a conformal map between any two simply connected domains in the complex plane, provided they are not the entire plane. The existence of this Riemann map is a priori not obvious: Even relatively simple Riemann mappings (for example a map from the interior of a circle to the interior of a square) have no explicit formula using only elementary functions.
Simply connected open sets in the plane can be highly complicated, for instance, the boundary can be a nowhere-differentiable fractal curve of infinite length, even if the set itself is bounded. One such example is the Koch curve. The fact that such a set can be mapped in an angle-preserving manner from the nice and regular unit disc seems counter-intuitive.
\bigskip
\begin{center}
\begin{tikzpicture}[scale=0.8]
    \pgfdeclarelindenmayersystem{Koch curve}{
        \rule{F -> F-F++F-F}
        }
        \draw [black, xshift=6.7cm]
    [l-system={Koch curve, step=1.4pt, angle=60, axiom=F++F++F, order=4}]
    lindenmayer system -- cycle;

    \pgfmathsetmacro{\kochwidth}{2pt*4*8}
    \draw [black] (2,1.15) circle (\kochwidth pt);
    \node at (2,1.15) {$\disk$};
    \node at (8.7,1.15) {$\Omega$};

    \draw [->, thick] (5,1.3) -- (6,1.3) node [midway, above] {$\psi$};
\end{tikzpicture}
\end{center}
\bigskip

We closely follow the proof by normal families in \cite{SteinShakarchi2003_complexanalysis}.

\subsubsection{Preliminary Results}
\begin{lemma}[Schwarz Lemma] \label{thm:SchwarzLemma}
    Let $f:\disk\to\disk$ be holomorphic with $f(0) = 0$. Then 
    \begin{enumerate}
    \item $|f(z)| \leq |z|$ for all $z \in \disk$.
    \item If for some $z_0\neq 0$ we have $f(z_0)=z_0$ then f is a rotation.
    \item $|f'(0)| \leq 1$ and if equality holds, then $f$ is a rotation.
    \end{enumerate} 
\end{lemma}
\textit{Proof.} See \cite{SteinShakarchi2003_complexanalysis} page 218.

\begin{definition}
    A family $\mathcal{F}$ of holomorphic functions on a domain $\Omega$ is called \textbf{normal} if every sequence in $\mathcal{F}$ contains a subsequence that converges uniformly on any compact subset of $\Omega$.
\end{definition}    
\begin{definition}
    A family $\mathcal{F}$ is called \textbf{uniformly bounded on compact subsets} of $\Omega$ if for every compact subset $K \subset \Omega$ there exists a constant $M_K$ such that $|f(z)| \leq M_K$ for all $z \in K$ and all $f \in \mathcal{F}$.
\end{definition}
\begin{definition}
    A family $\mathcal{F}$ of holomorphic functions on a domain $\Omega$ is called \textbf{equicontinuous} if for every $\epsilon > 0$ there exists a $\delta > 0$ such that if $z \in \Omega$ with $|z-z_0| < \delta$ we have $|f(z) - f(z_0)| < \epsilon$ for all $f \in \mathcal{F}$.
\end{definition}
\begin{theorem}[Montel]\label{thm:Montel}
    A family $\mathcal{F}$ of holomorphic functions on $\Omega$ that is uniformly bounded on compact subsets of $\Omega$ is normal if and only if it is equicontinuous on compacta.
\end{theorem}
\textit{Proof.} See \cite{SteinShakarchi2003_complexanalysis} page 225.

\begin{theorem}[Hurwicz]
    Let $\Omega \subset \mathbb{C}$ be a connected open subset and let $f_n: \Omega \to \mathbb{C}$ be a sequence of injective holomorphic functions that converges uniformly on compact subsets of $\Omega$ to a holomorphic function $f\neq 0$. 
    If $f$ has a zero of order $m$ at $z_0$ then for every $\eps>0$ and sufficientlly large $k=k(\eps)\in\N$ $f_k$ has precisely $m$ zeros in $B_{\eps}(z_0)$ including multiplicities. Moreover, the zeros converge to $z_0$ as $k\to\infty$. 
\end{theorem}
We will use the following corollary of Hurwicz' theorem in the proof of \ref{thm:RiemannMappingTheorem}:
\begin{corollary}[Uniform Convergence to Holomorphic Limit and Injectivity] \label{thm:HurwiczCor}
    Let $\Omega \subset \mathbb{C}$ be a connected open subset and let $f_n: \Omega \to \mathbb{C}$ be a sequence of injective holomorphic functions that converges uniformly on compact subsets of $\Omega$ to a holomorphic function $f$. Then $f$ is either constant or injective.
\end{corollary}
\red{\textit{Proof.?}}


\begin{proposition}[Cauchy Inequality] \label{thm:CauchyInequality}
    Let $f$ be holomorphic on an open set containing the closure of a ball $B_R(z_0)$ centered at $z_0$ of radius R. Then $$ |f^{(n)}(z_0)|\leq \frac{n! \lVert f \rVert _C}{R^n},$$ where $ \lVert f \rVert _C= \displaystyle \sup_{z\in C}|f(z)|$ on the boundary circle $C$.
\end{proposition}

\textit{Proof.} See \cite{SteinShakarchi2003_complexanalysis} page 48.

\begin{theorem} [Maximum Modulus Principle] \label{thm:MaximumModulusPrinciple}
    Let $\Omega\subset\C$ be a connected open bounded set and $f:\Omega\to\C$ holomorphic. If $z_0$ is a point in $\Omega$ such that $|f(z_0)| \geq |f(z)|$ for all $z$ in a neighborhood of $z_0$, then $f$ is constant on $\Omega$. 
\end{theorem}
\textit{Proof.} See \cite{SteinShakarchi2003_complexanalysis} page 92.

\begin{theorem}[Implicit Mapping Theorem] \label{thm:ImplicitMappingTheorem}
    Let $r > 0$ be a radius, and let $x_0 \in \mathbb{R}^n$, $y_0 \in \mathbb{R}^m$. Consider the open set $W = B_r(x_0) \times B_r(y_0) \subset \mathbb{R}^n \times \mathbb{R}^m$ defined as
$$
W = \{ (x, y) \in \mathbb{R}^n \times \mathbb{R}^m : \| x - x_0 \|_2 < r \text{ and } \| y - y_0 \|_2 < r \}.
$$
Let $F: W \to \mathbb{R}^m$ be a continuous function satisfying the following conditions:
\begin{enumerate}
    \item $F(x_0, y_0) = 0$.
    \item The partial derivatives $\partial_{y_k} F : W \to \mathbb{R}^m$ exist for all $k \in \{1, \dots, m\}$ and are continuous on $W$.
    \item The partial differential $D_y F(x_0, y_0)$ (the differential of the map $y \mapsto F(x_0, y)$ at $y_0$) is invertible.
\end{enumerate}
Then there exist radii $\alpha, \beta \in (0, r)$ such that for the open balls $U_0 = B_\alpha(x_0) \subset \mathbb{R}^n$ and $V_0 = B_\beta(y_0) \subset \mathbb{R}^m$, there exists a unique continuous function $f: U_0 \to V_0$ satisfying:
\begin{itemize}
    \item $f(x_0) = y_0$.
    \item $\forall (x, y) \in U_0 \times V_0: F(x, y) = 0 \quad \iff \quad y = f(x). $
\end{itemize}
\end{theorem}
\textit{Proof.} See \cite{EinsiedlerWieser2022AnalysisSkript} page 573.

\subsubsection{Proof of Riemann Mapping Theorem}
\textit{Step 1: Existence of a bounded injective holomorphic map to the unit disk} \\
Let $\Omega$ be a simply connected open proper subset of $\mathbb{C}$. We show that $\Omega$ is conformally equivalent to an open subset of the unit disk containing the origin. 
Indeed, choose $\alpha \notin \Omega$ and consider the function $$f(z) := log(z-\alpha)$$ on $\Omega$, which is well-defined and holomorphic since $z-\alpha$ never vanishes on $\Omega$. 
Note $f$ is injective since $e^{f(z)} = z-\alpha$ is ($f(z_1) = f(z_2) \implies z_1 - \alpha = z_2 - \alpha$).
Then for a point $\omega \in \Omega$ we get $f(z) \neq f(\omega) + 2\pi i \quad\forall z \in \Omega$ since otherwise we would find $z = \omega$ again by exponentiating.
In fact, $f(z)\cap B_{\epsilon}(f(\omega) + 2\pi i) = \emptyset$ for some $\epsilon > 0$ since otherwise we would find a sequence $z_n \to \omega$ with $f(z_n) \to f(\omega) + 2\pi i$, contradicting the continuity of $f$.
Finally, the function $$g(z) := \frac{1}{f(z) - (f(\omega) + 2\pi i)}$$ is well-defined, holomorphic and injective on $\Omega$ and maps $\Omega$ to a bounded subset $g(\Omega)\subset\mathbb{C}$, so $g$ is conformal. By boundedness of $g(z) < \frac{1}{\epsilon}$ we can scale and translate $g(\Omega)$ to contain the origin and fit into the unit disk.
\bigskip

\textit{Step 2: Normalization and non-zero derivative} \\
By step 1 we can assume $\Omega$ to be an open subset of the unit disk with $0\in\Omega$.
Consider the family $$\mathcal{F}:= \{ f: \Omega \xhookrightarrow{} \mathbb{D} | f(0) = 0\}$$ of all injective holomorphic functions which map the origin to itself.
Note that $\mathcal{F}\neq\emptyset$ since it contains the identity, and it is a uniformly bounded family by construction (maps into unit disk). 
In order to use the Hurwicz corollary and get injectivity, we now want to find a function $f \in \mathcal{F}$ that maximizes $|f'(0)|$ (to rule out constant functions yielded otherwise by the application of said corollary).
Observe that by the Cauchy inequality \ref{thm:CauchyInequality} $|f'(0)|$ are uniformly bounded for $f$ in $\mathcal{F}$.
Next, let $$s:= \displaystyle \sup_{f\in\mathcal{F}} |f'(0)|.$$ and choose a sequence {$f_n$}$\subset\mathcal{F}$ such that $|f_n'(0)| \to s$ as $n\to\infty$.
By Montel's theorem, {$f_n$} has a subsequence converging uniformly on compacta to a holomorphic $f$ on $\Omega$.
Since $s\geq 1$ (the identity is in $\mathcal{F}$), $f$ is non-constant. Hence by the Hurwicz corollary \ref{thm:HurwiczCor}, $f$ must be injective.
By continuity we have $|f(z)| \leq 1$ for all $z\in\Omega$, and since $f$ is non-constant, by the Maximum Modulus Principle \ref{thm:MaximumModulusPrinciple} $|f(z)|<1$.
Finally and since $f(0) = 0$, we have $f\in\mathcal{F}$ and $|f'(0)| = s$.
\bigskip

\textit{Step 3: Conformal mapping to the entire disk} \\
By injectivity it suffices to show $f$ is surjective.
Suppose towards a contradiction that $f$ is not surjective, we will construct a function in $\mathcal{F}$ with derivative greater than $s$ at the origin.
So let $\alpha\in\mathbb{D}$ be such that $\alpha\notin f(\Omega)$ and consider the automorphism of the unit disk that interchanges 0 and $\alpha$,
$$\psi_{\alpha}(z) := \frac{\alpha-z}{1-\bar{\alpha}z}.$$
Since $\Omega$ is simply connected and by continuity of $f$ and $\psi_{\alpha}(\Omega)$, the set $U:= (\psi_{\alpha}\circ f)(\Omega)$ is simply connected and does not contain the origin.
Thus we can define a square root function on $U$ by $$ q(w)=e^{\frac{1}{2}log w}.$$
Next, consider the function $$F=\psi_{q(\alpha)}\circ q \circ \psi_{\alpha}\circ f.$$
Then $F\in\mathcal{F}$ since $F(0) = 0$ and $F$ is holomorphic and injective since all the composing functions are. Also, $F$ maps into the unit disk since all the composing functions do.
But now if $h$ denotes the square function $h(w)=w^2$, then we must have $$f = \psi_{\alpha}^{-1}\circ h \circ \psi_{q(\alpha)}^{-1} \circ F = \Phi \circ F.$$
But $\Phi:\mathbb{D}\to\mathbb{D}$ satisfies $\Phi(0) = 0$ and is not injective since $F$ is but $h$ is not. 
By the Schwarz lemma \ref{thm:SchwarzLemma} we get $|\Phi'(0)| < 1$ and hence
$$|f'(0)| = |\Phi'(0)||F'(0)| < |F'(0)|,$$
contradicting maximality of $|f'(0)|$ in $\mathcal{F}$.

Finally, multiplying $f$ with a suitable unimodular complex number gives the desired conformal map from $\Omega$ to $\mathbb{D}$ with
\begin{equation}\label{eq:InitialConditionsOnPsi}
    \psi(0)=0 , \quad \psi'(0)>0.
\end{equation}
\red{REPLACE F BY PSI EVERYWHERE}
$\hfill\qed$

\begin{corollary}
    Any two simple connected open proper subsets of $\mathbb{C}$ are conformally equivalent.
\end{corollary}
\textit{Proof.} This follows directly from the Riemann mapping theorem by taking the unit disk as an intermediate step.
$\hfill\qed$

\begin{theorem}[Carathéodory] \label{thm:ExtensionToBoundary}
    The conformal mapping $\Phi:\disk\to \Omega$ can be extended to a continuous mapping $\Phi:\bar{\disk}\to\bar{\Omega}$ if the boundary $\Gamma=\del\Omega$ consists of a closed curve. \cite{Wegmann2005_num_methods_confmapping}, p. 357
\end{theorem}
\textit{Proof.} See \cite{Pommerenke1992_bdarybehaviour_confmaps} page 24.
\bigskip

Existence established, the problem now becomes the explicit construction of a conformal mapping.
Once the boundaries of the domain and target region are conformally mapped, the interior can be deduced from the boundary by the Cauchy Integral Formula. Hence why the main focus of the task lies on mapping boundary to boundary.

The strategy for numerical construction of the boundary mapping is to transform it into a boundary value problem.


% \red{add section 2.5:} error of operator K_N for methods based on function conjugation
\subsection{Reformulation as Boundary Value Problem}\label{chap:BoundaryValueProblems}\label{operatorK_N}

When the region $\Omega$ is bounded by a closed Jordan curve $\Gamma$ the mapping $\psi:\disk\to \Omega$ can be extended continuously to the closure $\overline{\disk}$ by Carathéodory's Extension Theorem \ref{thm:ExtensionToBoundary}. Then the boundary can be parametrized by a $2\pi$-periodic function $\eta$ in counterclockwise direction (positive orientation, aka \textit{normal representation} \cite{Wegmann2005_num_methods_confmapping}, page 387) and the mapping $\psi$ is determined by its boundary values 
\begin{equation} \label{eq:boundaryCorrespondence}
    \psi(e^{it})=\eta(S(t))
\end{equation}
for $t\in[0,2\pi)$ and $S$ the \textbf{boundary correspondence function}. 
By the implicit mapping theorem \ref{thm:ImplicitMappingTheorem}, the conformal mapping $\psi$ depends on the boundary curve of $\Omega$ by the boundary correspondence equation \ref{eq:boundaryCorrespondence}.

\red{a rh problem arises when considering the change of the conformal mapping under changes to the boundary curve. weg388}

From Definition \ref{def:ConformalMap} we know that a complex differentiable function with nonvanishing complex derivative is conformal, and complex differentiability in two dimensions can be caracterised by the Cauchy-Riemann equations (necessary and sufficient condition).
Denoting 
$$J:= \begin{pmatrix}
    \partial_x & -\partial_y \\
    \partial_y & \partial_x
\end{pmatrix}$$
we want to find a deformation $\psi= u+iv$ such that $J\psi = 0$.

As a byproduct of the Cauchy-Riemann equations, both components of a conformal map are harmonic functions, i.e. they satisfy Laplace's equation $\Delta u = 0 = \Delta v$ \red{PROOF}.
Thus, one way to construct conformal maps is to solve Laplace's equation with suitable boundary conditions.
% This formulation of the problem is called Riemann-Hilbert problem, and it is a special kind of boundary value problem.

\begin{definition}
    Two harmonic functions $u(x,y)$ and $v(x,y)$ on a domain are called \textbf{conjugate} if they satisfy the Cauchy-Riemann equations on the domain. E.g. $v$ is a conjugate for $u$, and $-u$ is a conjugate for $v$.
\end{definition}

This directly leads to the conclusion that it suffices to find only one of the components of $\psi=u+iv$, since the other one can then be deduced as its conjugate harmonic function.
This formulation is called the \textbf {Dirichlet problem} for the Laplace equation.
It can be described as finding $\psi$ analytic in $\disk$, continuous in $\overline{\disk}$ and satisfying 
\begin{equation} \label{eq:DirichletProblem}
    \begin{matrix}
        Re(\psi(e^{it}))=\eta(S(t)) &\text{ on }\del\disk \\
        \Delta u = 0 &\text{ in } \disk^{\circ},
    \end{matrix}
\end{equation}
where $\psi$ is $2\pi$-periodic and Hölder continuous.
This problem has a unique solution up to an imaginary constant, which can be constructed using the conjugation operator $$K\psi(s) := \frac{1}{2\pi}\int_{0}^{2\pi} \psi(t)cot(\frac{s-t}{2}) dt,$$ which is also known as Hilbert Transform (details on this in \cite{Castillo2025_hilberttransform}).
\red{
In principle, we have two options for finding conformal mappings, namely to solve linear or non-linear boundary value problems. The linear problems are naturally faster to solve numerically, however their \red{kernels} must be adapted to each new fixed boundary and thus these methods present complications when dealing with predefined boundaries. Non-linear methods on the other hand can be more flexible with respect to the boundary shape, but are computationally more expensive.
}


\subsection{Boundary Conditions} 

\begin{remark}
    Note that the Cauchy-Riemann equations do not guarantee a solution for arbitrary boundary data. A holomorphic map from the boundary of the unit disk onto some boundary of a convex set in $\C$ satisfies Cauchy-Riemann equations if the boundary of the target set can be described by non-negative Fourier frequencies \red{[PROOF]}. Thus, the choice of parametrization of the target region's boundary is usually another challenge posed when solving conformal mapping problems, but for the scope of this article we assume to be given a suitable parametrization of $\del\Omega$.
\end{remark}
\subsubsection{Dirichlet Boundary Conditions}
\subsubsection{Neumann Boundary Conditions}
% \subsection{Mixed boundary conditions (Wec 4.3)(?)}


In order to use numerical methods efficiently via the \red{Fast Fourier Transform def?} we will prefer the functions' Fourier series representations. Hence on a grid of $N = 2n$ equidistant \red{why? FFT requires it? has crowding as an effect...} points $t_j = \frac{(j-1)2\pi}{N}$, $\psi$ will be written as $$\psi(t_j)=\sum_{k=-n+1}^{n} c_k e^{ikt_j} \text{ for } j\in [N]$$ and the conjugation operator can be approximated by the operator $K_N$ defined as $$K_N\psi(t_j) = \sum_{k=1}^{n-1} -i c_k e^{ikt_j} + ic_{-k} e^{-ilt_j}.$$
The function $K_N$ is thus defined as a trigonometric polynomial obtained by interpolating $\psi$ at the grid points. This approximation of the actual operator $K$ satisfies $$\|K \psi - K_N \psi\|_{\infty} \in O(n^{-\alpha+1/2}) $$ for $\psi\in C^{\alpha}$ Hölder continuous.
If $\psi$ is smoother, e.g. $\psi \in C^{k-1}$, the error decreases to $$\|K \psi - K_N \psi\|_{\infty} \in O(n^{-k}log(n) \|\psi^{(k)}\|_{infty})$$ \cite{Wegmann2005_num_methods_confmapping} p. 362.
\red{add more details on this operator?}


\subsection{Solution of the Boundary Value Problem via Fredholm Integral Equations of the Second Kind}
We often numerically solve BVPs by transforming them into integral equations. Integral equations are equations in which an unknown function appears under an integral sign. 
They are classified into two main types: Fredholm and Volterra integral equations. Fredholm integral equations have fixed limits of integration, while Volterra integral equations have variable limits of integration. Both types can be further categorized into first kind and second kind, depending on whether the unknown function appears only under the integral sign or also outside of it.

\begin{definition}
    A Fredholm integral equation of the second kind is an equation of the form
    $$ f(x) = \lambda \int_{a}^{b} K(x,t) \phi(t) dt + g(x) $$
    where $f$ and $g$ are known functions, $\phi$ is the unknown function to be solved for, $K(x,t)$ is a given kernel function, and $\lambda$ is a constant.
\end{definition}

There are several methods to solve Fredholm integral equations of the second kind, which we will explore in what follows\dots
\subsubsection{Neumann Kernel}
\red{NEUMANN KERNEL?\cite{Gaier1964} used to solve fredholm integral eq of the second kind}

\subsubsection{Hilbert Transform}


\iffalse
\cite{Wechsung2019}: Mesh quality measures comment p 54
\fi

\subsection{Sobolev Spaces}
Let $L^2$ be the space of all $2\pi$-periodic complex functions $f$ which are square integrable over $[0, 2\pi]$ equipped with the inner product 
$$(f,g)_2= \frac{1}{2\pi}Re \int_{0}^{2\pi} f(f) \overline{g(t)} dt.$$
\begin{definition}
    The \textbf{Sobolev space $W$ }is defined as the space of all absolutely continuous functions $f\in L^2$ such that the derivative $f'$ exists and is also in $L^2$. The inner product on $W$ is defined as
    $$(f,g)_W = (f,g)_2 + (f',g')_2.$$
\end{definition}
This is a Hilbert space over $\R$. The subspaces of real functions are denoted $L_{\R}^2$ and $W_{\R}$ respectively. Note that we can decompose $W$ into the direct sum of the subspaces $W = W^{+} \oplus W^{-}$ where $f \in L^2$ is decomposed as follows into its Fourier series:
$$f(t) =  \sum_{n = -\infty}^{\infty} a_n e^{int} = 
    \underbrace{\sum_{-\infty}^{0} a_n e^{int} +i(\text{Im}(a_1))e^{int}}_{=:f^{-}\in W^{-}} + 
    \underbrace{(\text{Re}(a_1))e^{int} + \sum_{n=2}^{\infty} a_n e^{int}}_{=:f^{+}\in W^{+}}.
$$
Recall the definition of $\Phi$ as the mapping from the boundary of the disk \red{VIA ETA} and subject to \ref{eq:InitialConditionsOnPsi}. This can now be expressed as $\Phi \in W^{+}$\red{HUH}.
Then, since the boundary function \ref{eq:boundaryCorrespondence} maps the unit circle to $\Gamma$ there exists \red{why} a function $\hat{u}$ such that 
$$\Phi(t)=\eta(t+\hat{u}(t)) \quad \forall t.$$
By the implicit function theorem \ref{thm:ImplicitMappingTheorem} $\hat{u}$ is continuously differentiable, hence $\hat{u} \in W_{\R}$.
This tells us that the function $\Phi$ we are looking for lies in the intersection of a certain manifold $M:= \{u \in W_{\R}: \eta(t+u(t))\}$ with our space $W^{+}$. This formulation is used in chapter \ref{chap:AlternatingProjections}.


\subsection{Riemann-Hilbert Problems}
XXXXXXXXXXXXXX

\subsection{Complex Analysis/ Function Theoretic Tools}
\begin{definition}
    A \textbf{Möbius transform} is a function on the extended complex plane $\hat\C:=\C\cup\{\infty\}$ which is uniquely determined by where it sends three points. It has the form
    $$ f(z) = \frac{az + b}{cz + d} $$
    with complex numbers $a,b,c,d$ such that $ad - bc \neq 0$.
\end{definition}
More explicitly, given three distinct points $z_1,z_2,z_3 \in \C$ and three distinct points $w_1,w_2,w_3 \in \C$ there exists a unique Möbius transform $f$ such that $f(z_i) = w_i$ for $i=1,2,3$:
$$ f(z)=\frac{(z-z_1)(z_2-z_3)}{(z-z_3)(z_2-z_1)}$$
In particular, Möbius transforms are conformal and more general than affine maps, since they can e.g. map $\infty$ to $0$ and vice versa.


\subsection{Meshes}
In order to numerically solve boundary value problems the domain is usually discretized by creating a mesh $\M$ of $N$ nodes (points) and $M$ elements (cells) covering the domain $\Omega$. 

\subsection{Mesh Structure}
In order to keep the approximation error as low as possible while maintaining good solver performance, we typically want uniform meshes, e.g. where the triangulation is close to equilateral. The quality of a mesh is measured in terms of its individual cells where for a mesh $\mathcal{M}:= \{K\}$ of triangles $K$ such that $$\overline{\Omega}= \displaystyle\cup_{K\in\mathcal{M}}K$$ we define $d(K)$ the diameter of the smallest $K$-circumscribing ball (aka diameter of $K$) and $\mu(K)$ the diameter of the largest ball inscribed in K. Then a measure \cite{Wechsung2019_shapeopt_robustsolvers_incompressible} for the quality of K is the ratio of these diameters
$$ \rho(K) := \frac{d(K)}{\mu(K)} \in [1,\infty) .$$


\subsection{Crowding}
\cite{Banjai2008_SC_crowdingworkaround}
Crowding is the phenomenon that occurs when a set of more or less equally distributed points on the domain are mapped to a much more dense set of points on the target region, thereby causing numerical issues due to the angles on the grid becoming very small. This typically occurs when the target region is elongated, i.e. has a high aspect ratio. 
Wegmann \cite{Wegmann2005_num_methods_confmapping} proved the following result.

\begin{theorem}
    When the region $G$ can be enclosed in a rectangle with sides $a$ and $b$, $b \leq a$, such that $G$ touches both small sides then the conformal mapping $\phi : D \rightarrow G$ satisfies
    $$
    \|\phi'\|_D \geq b \psi(b/a)
    $$
    with a function $\psi(\tau)$ which behaves for small $\tau$ like
    $$
    \psi(\tau) \approx \frac{1}{2\pi \sqrt{\epsilon}} \exp\left(\frac{\pi}{2\tau}\right).
    $$
\end{theorem} 

\red{DeLillo has shown how crowding affects the accuracy of numerical computations.}
\subsubsection{The operator R}
\red{
In some conformal mapping methods, boundary value problems as follows occur,
$$ \psi(e^{it})=B(t) + A(t)U(t) $$
where $A, B: \mathbb{C}\to \mathbb{C}$ and $U:\mathbb{C}\to \mathbb{R}$. Multiplication with $\bar{A}$ yields the RH problem $$ Im(\bar{A(t)}\psi(e^{it})) = Im(\bar{A(t)}B(t))$$
This problem can be solved by the operator $R_{\beta}$, which is defined as follows:
}
This follows from the following theorem:
\begin{theorem}
    There exists a function $\psi$ analytic in $D$ with $\psi(0)= 0$ satisfying the boundary problem if and only if U is a solution of the Fredholm integral equation of the second kind
    $$(I+R_{\beta})U=g$$
    with the right-hand side
    $$g := -Re ( e ^{-i\beta} ( I - iK + J)/B).$$ 
    Where $K$ is the conjugation operator and J the averaging operator.
\end{theorem}
