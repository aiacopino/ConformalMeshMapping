\iffalse METRICS
- Numerical stability of different conformal mapping methods
- Boundary discretization requirements (how smooth does $\gamma$ need to be?)
- Mesh quality preservation - how does the mapping affect triangle quality?
- Computational complexity
- Input and output formats
\fi
\section{Existing Methods}
Let $G$ be a simply connected region in the complex plane with $0\in G$ and boundary $\gamma$. The Riemann mapping theorem guarantees the existence of a conformal map from the unit disk to $G$. Several numerical methods exist to construct such maps. This chapter aims to give an overview and compare the methods in terms of input/output format suitability/ boundary requirements, computational complexity, numerical stability and mesh quality preservation/ accuracy.
GIVEN MESH ON UNIT DISK AND BOUNDARY OF OMEGA, FIND PHI. 

\subsection{Potential Theoretic Methods}
% starting p 369 weg05

The accuracy of the conformal mapping depends to a large extent on the smoothness of the boundary curve $\gamma$. 

\subsubsection{Bergman Kernel Method}

\iffalse
\subsection{Mapping from the Region to the Disk}
\subsubsection{Extremum Principles}
% (WEG384)
Both the \color{red} Bergman and the Szegö norms \color{black} are very useful for characterizing the conformal mapping from G to a disk by extremum principles:
\begin{theorem}
    Let F be the conformal mapping from G to a disk normalized by $f(0)=0, f'(0)=1$.
    Then
    \begin{enumerate}
        \item (Principle of minimum area) F' is the unique function which minimizes $||f||_B$ among all functions $f \in B(G)$ satisfying $f (O) = 1$.
        \item (Principle of minimum length) $\sqrt{F'}$ is the unique function which minimizes $||f||_S$ among all functions $f \in S(G)$ satisfying $f (O) = 1$.
    \end{enumerate}
\end{theorem}

\subsubsection{Osculation Methods}
The osculation method (Schmiegungsverfahren) of Koebe [ 140] approximates F by a com-
position of elementary maps. It is universally applicable, since it requires no hypotheses at
all concerning the boundary 0G of the region G.
\subsubsection{Accuracy}
\fi 

\subsection{Mapping from the Disk to the Region}
When the region $G$ is bounded by a closed curve $\Gamma$ the mapping $\Phi:D\to G$ can be extended continuously to the closure $\bar{D}$ by Theorem \ref{thm:ExtensionToBoundary}. Then the boundary can be parametrized by a $2\pi$-periodic function $\eta(s)$ in counterclockwise direction (positive orientation, aka \textit{normal representation} \cite{Wegmann2005}, s. 387) and the mapping $\Phi$ is determined by its boundary values $$\Phi(e^{is})=\eta(s) \label{boundaryCorrespondence}$$ for $s\in[0,2\pi)$. 
By the implicit mapping theorem \ref{thm:ImplicitMappingTheorem}, the conformal mapping $\Phi$ depends on the boundary curve of $G$ by the boundary correspondence equation \ref{boundaryCorrespondence}.

\red{a rh problem arises when considering the change of the conformal mapping under changes to the boundary curve. weg388}


\subsubsection{Projection Methods}
Alternating Projection Method à la Neumann \cite{Wegmann2005} p389
\red{
Algo:
Start iteration with a $2\pi$-periodic function $S_0(t)$, e.g. $S_0(t)=t$.
Calculate Fourier coefficients $B_k$ of the boundary function using...
}
AP Method converges linearly if the boundary parametrization is 3-Hölder and the initial approximation $S_0$ is sufficiently close to the actual boundary correspondence function $S$.
Fourier calculation can be done very efficiently using FFT which makes AP one of the simplest and most robust methods for conformal mapping. However, it is also not very accurate for reasonably size grids, and converges very slowly for finer meshes.

\red{various improvements}
\subsubsection{Newton Methods}

\subsubsection{Interpolation}
\subsubsection{Accuracy}
Wegmann \cite{Wegmann2005} \red{section 2.5 } proved that accuracy of methods based on function conjugation is bounded by the error of the operator $K_N$ of the conjugation operator on the grid \ref{operatorK_N}.
Theodorsen's method is known \cite{Gaier1964} \red {add page/argument} to have an error of order $O(R^{-n})$ for regions bounded by analytic curves, where $R>1$ is the index of the largest disk centered at the origin to which the boundary parametrization $\eta$ can be extended as a conformal map, and $n$ is the order of the polynomials approximating $\eta$.
Wegmann compared the accuracies of the AP, OAP, Theodorsen and Wegmann methods for the mapping from the disk to an inverted ellipse \red{example 4?} and found that OAP is most efficient for low accuracy and Newton methods are best for high accuracy calculations \cite{Wegmann2005} p.415.\red{add figure?}
\red{Note the computational costs are mainly determined by the FFTs and this parameter is dependent on the number of grid points.}


\subsection{Schwarz-Christoffel Method}
$R_w = \mathbb{D}$ disk \\
$B_w = \del\mathbb {D}$ boundary of D\\
$R_z = \Omega$ = polygon \\

One class of methods for finding the conformal mapping $\Phi$ is given by the Schwarz-Christoffel equation, which relates the derivative of $\Phi$ to an integral over the boundary of the target domain $\Omega$ when $\Omega$ is a polygon.
\subsubsection{Preliminaries and Notation}
\begin{definition}
    A \textbf{Polygon} is a planar figure whose boundary is made of a chain of connected line segments which we will call arcs, connecting corner points.
\end{definition} 

The unit disk $\mathbb{D}$ is in particular a polygon, and we parametrize the boundaries $\del\mathbb{D}$ and $\del\Omega=\Gamma$ by collections of arcs $s_{\mathbb{D}}$ and $s_{\Omega}$ respectively, in positive mathematical orientation. 
These arcs being smooth yields tangents with well-defined derivatives at every point of the boundary curves except for corners, and we denote the angles of these tangents with $\theta_{\mathbb{D}}(s_{\mathbb{D}})$ and $\theta_{\Omega}(s_{\Omega})$ respectively. If $z_0$ is a corner it will have a turning angle of 
$$\measuredangle\theta_{\mathbb{D}}(z_0)=\theta_{\mathbb{D}}(z_0+\eps)-\theta_{\mathbb{D}}(z_0-\eps),$$ 
where $\eps\to0$. The same applies to corners of $\Omega$. Then,
$\frac{\del\theta_{\mathbb{D}}}{\del s_{\mathbb{D}}} = \measuredangle\theta_{\mathbb{D}}(z_0)\delta(s_{\mathbb{D}}-z_0)$ for $\delta$ the Dirac function, and similarly for $\theta_{\Omega}$ and thus $\theta_{\mathbb{D}}$ and $\theta_{\Omega}$ are piecewise continuously differentiable functions with jump discontinuities at the corner points.

\red{\subsubsection{the log derivative of Phi}}

\subsubsection{Green's Functions}
\begin{definition} 
    A Green's function is generally a concept for solving differential equations containing a linear operator and some initial and boundary conditions.
\end{definition}
In the case of the SCE the Green's function are defined as $G(z,z'):\mathbb{D}\to\mathbb{R}$ satisfying $$\nabla^2 G(z,z') = 2\pi \delta(u-u')\delta(v-v') $$ inside the disk and $$\frac{\del G(z_B, z')}{\del n} = \beta_i$$ where $n$ is the outward normal vector at the boundary point $z_B \in \del\mathbb{D}$ and $\beta_i$ is a real constant associated with the $i$'th arc's length $l_i$ such that $\sum l_i\beta_i = 2\pi$. It can be shown according to Floryan and Zemach that this defines a unique Green's function up to an additive constant \cite{FLORYAN1987}.
The SCE can be written explicitly whenever the Green's function is \red{obtainable in closed analytic form, i.e. expressable via a finite number of elementary operations (is that what is meant on p348? i looked up wikipedia for closed analytic form)}.

\subsubsection{Schwarz-Christoffel Equation Variants}
Let $\mathcal{G}(z, z_B')$ be a complex extension of the above defined $G(z,z_B')$, i.e. $\mathcal{G}(z, z_B')$ is analytic with real part $G(z,z_B')$.
A Schwarz-Christoffel equation for a conformal mapping $\Phi(z):\mathbb{D}\to\Omega$ has the form $$ log \frac{d\Phi}{dz}= C+\displaystyle\sum_i Q_i,$$ where $C\in\mathbb{C}$ is a constant and the $Q_i$ are the Green's function integrals over the boundary arcs of $\mathbb{D}$.
Some parameters and constants have to be determined in order to get a unique $\Phi$ for a particular given $\Omega$. The Riemann mapping theorem allows \red{HOW?} for the first three real parameters to be preassigned, for example to three boundary points in the case of simply connected bounded $\Omega$. The remaining degrees of freedom must be solved for using properties of the arcs $s_{\Omega}(s_z)$ and $s_{\mathbb{D}}(s_z)$. 

The most general form is called Schwarz-Christoffel equation with subtraction \cite{FLORYAN1987}:
$$log(\frac{d\Phi}{dz}) = C_0 - \frac{1}{2\pi}\displaystyle\int_{\del\mathbb {D}} [\mathcal{G}(z, z_B')-\mathcal{G}(z_0, z_B')] \times [d\theta_{\Omega}(s_z')-d\theta_{\del\mathbb{D}}(s_z')]$$
where $C_0\in\mathbb{C}$ is a constant, $z_0$ is a fixed point in $\mathbb{D}$\red{or del?}, $\mathcal{G}(z,z_B')$ is the fundamental solution of the Laplace equation in $\del\mathbb{D}$ with singularity at $z_B' \in \del\mathbb{D}$\red{not at the boundary of omega?}. However, we can use an unsubtracted form of this equation since the integrals converge separately and we know our $\Omega$ is bounded (hence we need not care about behaviours at infinity), yielding the simpler form
$$log(\frac{d\Phi}{dz}) = C - \frac{1}{2\pi}\int_{\del\mathbb{D}} \mathcal{G}(z, z_B') [d\theta_{\Omega}(s_z')-d\theta_{\del\mathbb{D}}(s_z')]$$



\red{Alternative approaches to numerical conformal mapping are enumerated in Henrici’s book [8].}