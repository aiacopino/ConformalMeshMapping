\usepackage{amsmath, amsthm, amssymb, amsfonts, mathtools}
\usepackage{bibentry}

% \usepackage{../code/algorithmicx/algpseudocode}
%\usepackage{algpseudocode}
\usepackage{algorithmic}
%\usepackage{algorithmicx}
\usepackage{algorithm}

\usepackage{rotating, booktabs, tabularx, pdflscape, makecell} % table
\usepackage{array}

\usepackage{url} % bibtex wikipedia

\usepackage{tikz, xcolor, tikz-cd}
\usetikzlibrary{lindenmayersystems} % fractals
\usetikzlibrary{shapes.geometric, shapes.misc, 
                arrows, positioning, angles, quotes, calc, 
                decorations.markings, bending, intersections}

% \usepackage{standalone}
% definitions/ formatting for flowchart
\tikzstyle{startend} = [ellipse, minimum width=3cm, minimum height=1cm, text centered, draw=black]
\tikzstyle{process} = [rectangle, minimum width=3cm, minimum height=1cm, text centered, draw=black]
\tikzstyle{decision} = [diamond, minimum width=3cm, minimum height=1cm, text centered, draw=black, aspect=2]
\tikzstyle{arrow} = [thick, ->, >=stealth]


\newcommand{\del}{\partial}
\newcommand{\eps}{\varepsilon}
\newcommand{\disk}{\mathbb{D}}
\newcommand{\R}{\mathbb{R}}
\newcommand{\C}{\mathbb{C}}
\newcommand{\N}{\mathbb{N}}
\newcommand{\M}{\mathcal{M}}
\newcommand{\red}[1]{\textcolor{red}{#1}}
\newcommand{\green}[1]{\textcolor{green}{#1}}
\newcommand{\bigO}{\mathcal{O}}

\newtheoremstyle{custom}{3mm}{3mm}{\normalfont}{0pt}{\bfseries}{}{\newline}{}
\theoremstyle{custom}

%\newcommand{\mydef}[1]{\indent\textbf{Definition.}\newline #1\par}
\newtheorem{definition}{Definition}
\newtheorem{theorem}{Theorem}[section]
\newtheorem{lemma}[theorem]{Lemma}
\newtheorem{corollary}[theorem]{Corollary}
\newtheorem{proposition}[theorem]{Proposition}
\newtheorem{remark}{Remark}