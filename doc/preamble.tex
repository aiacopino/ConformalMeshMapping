\usepackage{amsmath, amsthm, amssymb, mathtools}
\usepackage{bibentry}

% \usepackage{../code/algorithmicx/algpseudocode}
% \usepackage{algpseudocodex}
\usepackage{algorithmic}
\usepackage{algorithm}

\usepackage{tikz, xcolor}
\usetikzlibrary{lindenmayersystems}

\newcommand{\del}{\partial}
\newcommand{\eps}{\varepsilon}
\newcommand{\disk}{\mathbb{D}}
\newcommand{\R}{\mathbb{R}}
\newcommand{\C}{\mathbb{C}}
\newcommand{\N}{\mathbb{N}}
\newcommand{\red}[1]{\textcolor{red}{#1}}
\newcommand{\green}[1]{\textcolor{green}{#1}}
\newcommand{\bigO}{\mathcal{O}}

\newtheoremstyle{custom}{3mm}{3mm}{\normalfont}{0pt}{\bfseries}{}{\newline}{}
\theoremstyle{custom}

%\newcommand{\mydef}[1]{\indent\textbf{Definition.}\newline #1\par}
\newtheorem{definition}{Definition}
\newtheorem{theorem}{Theorem}[section]
\newtheorem{lemma}[theorem]{Lemma}
\newtheorem{corollary}[theorem]{Corollary}
\newtheorem{proposition}[theorem]{Proposition}
\newtheorem{remark}{Remark}